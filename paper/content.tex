\section{Introduction}

% short motivation
% very broad overview of the procedure
% list used tools

\section{Methods}
% detailed approaches
% include error evaluation
Our implemented procedure contains the following steps:
\begin{itemize}
    \item Photo recording for camera calibration. The RGBD - camera is mounted onto the robot endeffector and a checkerboard of known dimensions is placed at the working table. The robot is then moved through a number of different positions. At each position, RGB and infrared photos are taken and stored with the current robot configuration.
    \item Calculation of camera calibration and hand-eye calibration. Using an OpenCV procedure, corners of the checkerboard in both infrared and RGB-images are detected. A standard camera calibration applied for each corners from the respective camera gives the intrinsic camera matrices and estimates of 3D-corner points of the checkerboard. A stereo calibration is now applied to yield the relative transformation between the RGB and infrared cameras. The estimates of the 3D corner points are then used to calculate the transformation from the endeffector to the camera.
    \item Phantom Recording. The checkerboard is replaced with the phantom body model at the workplace. The robot now traverses a set of positions to capture the target ball. At each position, a colored pointcloud generated by the camera driver is stored with its current position.
    \item Target estimation. Each pointcloud is transformed into base coordinates using the camera-endeffector transformation and robot pose. The points are then filtered by color comparison with statistical outliers being removed to obtain a set of ball points.
    \item Needle insertion. A straight cartesian path from the current robot position to the estimated center of the ball is taken and fine-sampled. For each point, close joint configurations are calculated and a quintic polynomial spline is used to connect the fine-sampled positions in the configuration space. The needle is then attached to the robot endeffector and moved using the quintic spline path.
\end{itemize}

\subsection{Kinematics}
In order to obtain transformation matrices and calculate joint positions from either given target points or joint position, a kinematics calculation is done. 
The forward kinematics is computed using the so-called modified Denavit-Hartenberg parameters provided by Franka Emika \cite{panda-spec}. For each joint, the total transformation matrix is given by 
\begin{equation}
    {}^{n - 1}T_n  = \operatorname{Rot}_{x_{n-1}}(\alpha_{n-1}) \cdot \operatorname{Trans}_{x_{n-1}}(a_{n-1}) \cdot \operatorname{Rot}_{z_{n}}(\theta_n) \cdot \operatorname{Trans}_{z_{n}}(d_n) \\
\end{equation}
which evaluates to 
$$
\left[
\begin{array}{ccc|c}
    \cos\theta_n & -\sin\theta_n  & 0 & a_{n-1} \\
    \sin\theta_n \cos\alpha_{n-1} & \cos\theta_n \cos\alpha_{n-1} & -\sin\alpha_{n-1} & -d_n \sin\alpha_{n-1} \\
    \sin\theta_n\sin\alpha_{n-1} & \cos\theta_n \sin\alpha_{n-1} & \cos\alpha_{n-1} & d_n \cos\alpha_{n-1} \\
    \hline
    0 & 0 & 0 & 1
  \end{array}
\right]
$$
\cite{wiki-denavit-hartenberg}.
To determine the homogeneous endeffector transformation, each transformation matrix is constructed and multiplied in order using Numpy. 

The Panda robot has 7 controllable joint angles, which allows for a multitude of possible angles for most possible configurations. Our inverse kinematics uses an iterative algorithm (c.f. incremental inverse kinematics) to minimize the squared error starting from a possible configuration. 
We used the SciPy sequential least squares programming (SLSQP)-algorithm for constrained quadratic minimization, using interval bounds for the robot configuration space given in \cite{panda-spec}. The error function to be minimized is given by
$$
{|{}_0T^7 - T_{\text{target}}|_F}^2
$$, where the Frobenius norm is defined by ${|T|_F}^2 := \sum_{i,j} T_{ij}^2$. For gradient calculation, the algorithm uses numeric first-order approximation.

\subsection{Trajectory Generation}
For normal movement between positions, we use a combined trajectory consisting of two quintic polynomial segments with a linear segment in between. The panda robot allows only trajectories, which are approximately continuous in position, velocity, acceleration. This enforcement is computed using limits in the approximated velocity, acceleration and jerk \cite{panda-spec}.
The first quintic section acts to accelerate the robot to the desired speed of the joint in the linear segment. The robot is then decelerated to zero velocity.
To estimate the time of the segment, the critical (i.e., duration determining) joint determines the motion, i.e.
$$
    T_{\text{estimated}} = \operatorname{arg}_i(\frac{\delta q_i}{\dot{q}_{i, \text{max}}}).
$$ where $\dot{q}_{i, \text{max}}$ denotes the maximum speed of the joint according to \cite{panda-spec} and $Δqᵢ$ refers to the configuration distance of joint $i$.
For this $T_{\text{estimated}}$, linear joint speeds are calculated. The first quintic polynomial $p₁$ is now calculated to satisfy
\begin{align*}
    p_1(0) &= q_0 \\
    \dot{p}_1(0) &= 0 \\
    \ddot{p}_1(0) &= 0 \\
    \Ddot{p}_1(0) &= 0 \\
    \dot{p}_1(T_{\text{speedup}}) &= v_{\text{linear}} \\
    \ddot{p}_1(T_{\text{speedup}}) &= 0
\end{align*}
where $T_{\text{speedup}}$ is estimated to fit the maximum acceleration.
% todo

\subsection{Needle Insertion}
\input{sections/calibration}
\input{sections/pclprocessing}

\section{Application structure}

\section{Conclusion}

